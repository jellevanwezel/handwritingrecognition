%!TEX root = ../hr-paper.tex
\section{Introduction} % (fold)
\label{sec:introduction}


% --------------------------
% Introduction to the field

% why is charecter recognition it important?

% 	-numberplate
% 	-text translation
% 	-labels on products
% 	-digitalizing of old texts

% Why segmentation is important

%	- 3 parts of the system
%	- segmentation 
% 		- Noise reduction
% 		- Finding characters

% Focus of this paper.

% what methods have been used in the past?

% 	- Vertical and horizontal projection based
% 	- Recognition based
% 	- Scelaton based
% 	- Connected component based


% what are we going to do?

% 	- introduction to the dataset
% 	- connected components in combination with projection.
% 	- our segmented chars are used by our 2 team mates.

% what does this contribute to the field?

% 	- A pragmatic way of recognition on the used and other datasets
% 	- (research question) A comparison with our results to other people their results (depends if we can get other results, 2 different datasets...)
% 	- (hypothesis) Its probably oke for the time commited and the amount of work put in to it.

%  (The entire introduction should logically end at the research question and thesis statement or hypothesis.)


% 	- how the paper will continue.
% 		dataset
% 		method
% 		results
% 		conclusion

% 	research question 	- How does the implemented segmentation method perform (in comparison with existing methods? / methods of other groups)
% 	hypothesis			- Its probably oke for the time commited and the amount of work put in to it.


% --------------------------


Character recognition is an important part in the fields of computer vision and artificial intelligence. For a machine to recognize sentences, words ,and characters in a given image is useful for multiple goals. For example to digitize handwritten texts. digitized texts have a number of advantages over physical texts. An important advantage is that digital texts are more easily distributed and reproduced. It is also easier to search in digital texts. But probably the greatest advantage is storage. It is easier and takes less room to save and backup a few bytes on a disk than it is to physically backup archive the same texts. As is often the case in the fields computer vision and artificial intelligence, compression is the end goal.

An other practical application of character recognition is text translation. Text recognition for the use of translation is useful for example tourists who use their smart-phone to take a picture of a sign and have that sign translated in their native language by an translation app. Or even better, have their text translated and inserted into the camera feed directly. This form of augmented reality is still in its infantsy but being able to recognize and digitize texts on walls, signs ,and posters for all kinds of purposes is not that hard to imagine. 

Usage of character recognition can also be imagined in the retail industry. Here it can be used for label recognition on certain products, currency ,and identification purposes for restricted product groups (like liquor and tobacco).

The automotive industry uses character recognition in their autonomous cars. They read speed limits and other road signs to form a state of awareness of their surroundings They abstract what the next course of action should be based partly on that gathered state. A normal road sign is not a character but a computer does not know the difference between `human' characters and a specific image. The important part is that the goal is to find meaning in an image by `reading' parts of that image. Sign recognition and character recognition are therefor closely related. They differ in that characters tend to come in a sequence forming sentences. Where road signs are mostly isolated.

Staying on the road character recognition is used for numberplate recognition. It is used by law enforcers to find stolen or unregister vehicles but also as an automated form of speed control. Here a mounted or portable camera is used to take pictures of cars with their number plates and the software recognizes the plates and the characters on them to identify the vehicle. 




% section introduction (end)
