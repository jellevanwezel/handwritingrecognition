%!TEX root = ../hr-paper.tex
\section{Method} % (fold)
\label{sec:method}

\todo[inline]{TODO: Small introduction to this section}

\todo[inline]{TODO: the skew of the chars is not done by this algorithm}

\subsection{Segmentation}


The Chinese characters were supplied in greyscale strokes of characters in the western orientation (from left to right).

%Otsu, N., "A Threshold Selection Method from Gray-Level Histograms," IEEE Transactions on Systems, Man, and Cybernetics, Vol. 9, No. 1, 1979, pp. 62-66.
In order to segment the characters from an image a combination of methods is used. First the image is binarized. Binarization is done by applying a Gaussian filter on the image to remove some of the noise. Then Otsu's method of finding a threshold is used. The image is then binarized using this threshold.

\todo[inline]{TODO: Add references for Otsu and gausian filtering}

After binarization the image is rotated. This is done because most images in the dataset are titled slightly. This is probably due to some variations in the orientation of the paper during the scanning process. Rotation is done by resizing the image in the vertical direction to half its size. The idea behind this is that the characters will be squeezed together and form a horizontal line this line can then be used to find the optimal rotation. this is done by taking the horizontal densities. The maximum densities are saved for the different rotations (between $+1\degree$ and $-1\degree$). The rotation with the largest maximum density is where the line is likely to be the most horizontal. This is taken as the optimal rotation. 


After the characters are rotated the vertical location of the characters are located by taking the horizontal densities. The densities are smoothed by using a averaging filter with ten of the neighboring densities. Then the densities are thresholded. The threshold is determined by taking $0.1 \times imageWidth$. From the resulting vertical areas the largest area is taken. This is based on the assumption that characters will produce the most wide peak in horizontal densities. For example many of the images hold a horizontal line above or beneath the characters. This line will produce a narrow but high peak when the horizontal densities are taken. The characters will in this case produce a lower and wider peak. After the determination of the vertical location of the characters the rest of the image is removed leaving only a horizontal stroke of characters.

\todo[inline]{TODO: Add image of the densities}


We are left with a isolated stroke of characters. In order to find the characters in this stroke the connected components are taken. This will give us a list of all the connected components in the image. If multiple components are above each other we assume that these components are part of the same character. These components are merged. Some components are not directly on top of each-other. It is determent how much the components overlap vertically. If the overlapping width is $0.4$ times the width of the width of the smaller of the two components, they are merged. 

After merging vertically we are left with a list of components that might be outliers in terms of width in comparison with the mean width of the characters in the supplied labeled dataset. In order to recognize outliers the mean and standard deviation was calculated from the characters in the labeled dataset. A component is deemed a small out-lier if:

\begin{equation}
C_w < \mu_w + 2 \sigma_w 
\end{equation}

and big if:

\begin{equation}
C_w > \mu_w + 2 \sigma_w 
\end{equation}

Here $C_w$ is the component's width, $\mu_w$ is the mean width ,and $\sigma_w$ is the standard deviation in width.


Small components that are characters on their own are often preceded and followed by more white space than components that form a character with the component left or right from it. The small characters are inspected on this feature and it is determined if this small component is actually a character. Some small components are just noise. If a component is too small, has too much black pixels or too much white pixels, it is recognized as noise and it is removed from the list of components. 

\todo[inline]{TODO: At the moment these determinations are made by chosen thresholds. But these thresholds can be extracted from the labeled dataset}

All that is left in the list of small components are expected to be components of characters. The components are merged with their left or right neighbor based on the width the newly merged component will have. Small components merge with the component (left or right) that together will have the smallest width. If however both neighboring components are deemed as big by the outliers detection, no merging will take place. The component is measured after merging. If the merged components is still small, the algorithm will try to merge it again with its left or right neighbor.

\todo[inline]{TODO: Sometimes 2 characters are seen as one because of some noise above the characters or because they are both small. This can be detected by taking the components that are deemed big and see if they can be separated in two or more components if there is a lot of white space in between them or by looking at the vertical densities.}


\subsection{Feature Extraction}

\subsection{Classification}

% section method (end
